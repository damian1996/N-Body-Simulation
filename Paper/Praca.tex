\documentclass[13pt,twoside,a4paper]{article}
\usepackage{polski}
\usepackage[utf8]{inputenc}
\usepackage{amsmath}
\usepackage{amsfonts}
\usepackage{amssymb}
\usepackage{blindtext}
\usepackage{caption}
\usepackage{geometry}
\usepackage{graphicx}
\usepackage{listings}
\usepackage{color}
\usepackage{fontenc}
\usepackage{bm}
\usepackage{pstricks}
\usepackage{auto-pst-pdf}
\usepackage{hyperref}
\usepackage[
backend=biber,
style=alphabetic,
]{biblatex}
\addbibresource{biblio.bib}
\hypersetup{
  colorlinks, linkcolor=black
}
\DeclareCaptionFont{white}{\color{white}}
\DeclareCaptionFormat{listing}{\colorbox{gray}{\parbox{\textwidth}{#1#2#3}}}
\captionsetup[lstlisting]{format=listing,labelfont=white,textfont=white}


%\geometry{left=2.5cm,right=2.5cm,top=10.0cm,bottom=2.5cm}
\begin{document}

\begin{titlepage}
	\newcommand{\HRule}{\rule{\linewidth}{0.5mm}}
	
	\center
	
	\textsc{\Large Uniwersytet  Jagielloński\\
			 Wydział Matematyki i Informatyki\\
    		 Zespół Katedr i Zakładów Informatyki Matematycznej}\\[1.5cm]
	
	\textsc{\Large}\\[0.5cm]
	
	\textsc{\Large}\\[0.5cm]
	
	\HRule\\[0.4cm]
		
	{\huge\bfseries Wielowątkowa symulacja N ciał z implementacją w architekturze CUDA}\\[0.4cm] % Title 
	
	\HRule\\[1.5cm]
	
	\begin{minipage}{0.4\textwidth}
		\begin{flushleft}
			\large
			\textit{Autor}\\
			Damian \textsc{Stachura} % Your name
		\end{flushleft}
	\end{minipage}
	~
	\begin{minipage}{0.4\textwidth}
		\begin{flushright}
			\large
			\textit{Opiekun}\\
			dr Piotr \textsc{Danilewski} % Supervisor's name
		\end{flushright}
	\end{minipage}
	
	\vfill\vfill % Position the date 3/4 down the remaining page
	
	{\large\today} % Date, change the \today to a set date if you want to be precise
	
\end{titlepage}

\newpage
\tableofcontents

\newpage

\section{\LARGE Wstęp}
\bigskip
\bigskip

\subsection{\Large Przedstawienie problemu symulacji N ciał}



Symulacja N ciał polega na wyznaczeniu toru ruchów wszystkich ciał danego układu o danych masach, prędkościach i położeniach początkowych w oparciu o prawa ruchu i założenie, że ciała oddziałują ze sobą zgodnie z prawem grawitacji Newtona. \\

Problem jest rostrzygnięty dla co najwyżej dwóch ciał podlegających prawom klasycznej dynamiki Newtona i przyciągających się zgodnie z newtonowskim prawem powszechnego ciążenia. Ruch ten wygląda w ten sposób, że obiekty poruszają się po krzywych stożkowych, przy czym rodzaj krzywej zależy od całkowitej energii układu. Jeżeli energia jest na tyle mała, że ciała tworzą stan związany, czyli nie mogą się od siebie uwolnić, tylko muszą krążyć wokół siebie, to robią to po elipsach. Jeżeli energia układu jest wystarczająco duża, żeby mogły oddalić się od siebie dowolnie daleko, to najpierw zbliżają się do siebie, a potem uciekają na ogół po hiperbolach, chyba że energia będzie dokładnie na granicy pomiędzy stanem związanym a niezwiązanym, kiedy to będą poruszać się po parabolach.\\
$http://www.deltami.edu.pl/temat/fizyka/mechanika/2015/11/26/Problem_dwoch_cial/$

Zajmijmy się stanem związanym. Ruch po elipsach jest okresowy, co oznacza, że co pewien ustalony okres ciała zajmują te same położenia w przestrzeni. Możemy wyobrazić sobie, że te dwa ciała to np. Ziemia i Słońce albo układ podwójny gwiazd. Jeżeli tylko ciała te mają symetrię sferyczną, to ich ruch jest dokładnie taki sam, jak ruch punktów materialnych o tych samych masach. Teoria Newtona mówi nam, że te dwa ciała będą krążyć po okresowych orbitach dowolnie długo.\\

Problem trzech ciał jest rozstrzygnięty jedynie dla szczególnych przypadków, w których zakłada się na przykład, że masa jednego z ciał jest zaniedbywalnie mała. Zagadnienie to – tak zwany ograniczony problem trzech ciał – było po raz pierwszy postawione i częściowo rozwiązane przez Lagrange'a w drugiej połowie XVIII wieku. Badał on układ Słońce-Ziemia-Księżyc, w którym na dodatek można przyjąć, że jedno z masywnych ciał krąży wokół trzeciego po orbicie kołowej. 


\newpage

\begin{figure}
    \centering
    \def\svgwidth{\columnwidth}
    \input{draw.pdf_tex}
    \caption{Symulacja dwóch ciał po elipsie}
\end{figure}


\subsection{\Large Zastosowania}

 N-body simulations are widely used tools in astrophysics, from investigating the dynamics of few-body systems like the Earth-Moon-Sun system to understanding the evolution of the large-scale structure of the universe.[1] In physical cosmology, N-body simulations are used to study processes of non-linear structure formation such as galaxy filaments and galaxy halos from the influence of dark matter. Direct N-body simulations are used to study the dynamical evolution of star clusters. Common examples include a satellite orbiting a planet, a planet orbiting a star, two stars orbiting each other (a binary star), and a classical electron orbiting an atomic nucleus (although to solve the electron/nucleus 2-body system correctly a quantum mechanical approach must be used).

\subsection{\Large Wykorzystanie CUDA w implementacji}


\section{\LARGE Pierwsze podejście}
\bigskip
\bigskip

adadadadad

\subsection{\Large Jednowątkowy naiwny algorytm z pseudokodem}

\lstset{backgroundcolor=\color{gray}}
\begin{lstlisting}[frame=single, framerule=2pt, caption=compute next step]
void StepNaive::compute(tf3& positions, float dt) {
  std::fill(forces.begin(), forces.end(), 0);
  for(unsigned i=0; i<N; i++) {
    for(unsigned j=0; j<N; j++) {
      float distX = positions[j*3] - positions[i*3];
      float distY = positions[j*3+1] - positions[i*3+1];
      if(i!=j && fabs(distX) > 1e-10 && fabs(distY) > 1e-10) {
        float F = G*(weights[i]*weights[j]);
        forces[i*3] += F*distX/(distX*distX+distY*distY);
        forces[i*3+1] += F*distY/(distX*distX+distY*distY);
      }
    }
  }
  for(unsigned i=0; i<N; i++) {
    for(int j=0; j<2; j++) { // x, y
      float acceleration = forces[i*3+j]/weights[i];
      positions[i*3+j] += velocities[i*3+j]*dt + acceleration*dt*dt/2;
      velocities[i*3+j] += acceleration*dt;
    }
  }
}

\end{lstlisting}

\subsection{\Large Paralelizacja naiwnego algorytmu}

asasasa

\subsection{\Large Implementacja}

asasasa

\section{\LARGE Drugie podejście}

aaaaaaaaaaaaaaa

\subsection{\Large Algorytm Barnesa-Huta z pseudokodem}

aaaaaaaaaaaaaa

\subsection{\Large Zrównoleglenie algorytmu Barnesa-Huta}

Huuuuut

\subsection{\Large Implementacja}

Klepu klepu klep
mammaamadmadadada
Dummy text\\
apt-get install texlive-lang-polish\\
Random citation embeddeed in text.s\\
 sudo apt-get install texlive-bibtex-extra\\
 sudo apt-get install texlive-bibtex-extra biber\\
biber Praca\\
\url{https://www.sharelatex.com/learn/Bibliography_management_with_biblatex}\\
inkscape -D -z --file=drawing.svg --export-pdf=draw.pdf --export-latex


\section{\LARGE Podsumowanie}

\newpage

\nocite{*}
\printbibliography

\end{document}
